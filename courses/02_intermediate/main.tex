% !TeX program = xelatex

% Create command `createHandout` only if it doesn't already exists.
% To compile in presentation mode, pass a command createHandout doing nothing.
\providecommand{\createHandout}{handout}

\documentclass[
    aspectratio=169,
    \createHandout
]{beamer}

\usepackage{minted}
\usepackage{xcolor}
\usepackage{tcolorbox}
\usepackage{graphicx}
\usepackage{fontspec}
\usepackage{tabularray}

% allow to look for themes in another folder
% https://tex.stackexchange.com/a/284157/301699
\makeatletter
\def\beamer@calltheme#1#2#3{%
    \def\beamer@themelist{#2}
    \@for\beamer@themename:=\beamer@themelist\do
    {\usepackage[{#1}]{\beamer@themelocation/#3\beamer@themename}}}

\def\usefolder#1{
    \def\beamer@themelocation{#1}
}
\def\beamer@themelocation{}
\makeatother

\usefolder{..}
\usetheme{cexa-kokkos}

\AtBeginSection{
    \begin{frame}{Outline}
        \tableofcontents[currentsection, hideothersubsections]
    \end{frame}
}

\AtBeginSubsection{
    \begin{frame}{Outline}
        \tableofcontents[currentsection, currentsubsection]
    \end{frame}
}

\setminted{
    autogobble,
    fontsize=\small,
    bgcolor=lightgray,
    xleftmargin=0.5em,
    xrightmargin=0.5em,
    breaklines,
}

\usemintedstyle[text]{vim}
\setminted[text]{
    bgcolor=lightmain,
    fontsize=\scriptsize,
}

\NewTblrTheme{kokkostable}{
    \SetTblrInner{
        width=\linewidth,
        rowhead=1,
        rows={ht=\baselineskip},
        row{odd}={bg=lightgray},
        row{1}={bg=lightmain},
    }
}

\graphicspath{{../../images/}}

%Information to be included in the title page:
\title{Kokkos intermediate course}
\author{The CExA team}
\institute{CEA}
\date{\today}
\titlegraphic{%
    \includegraphics[height=4em]{kokkos.png}%
    \hspace{1em}%
    \includegraphics[height=4em]{cexa_logo.png}
}

% _____________________________________________________________________________

\begin{document}

\begin{frame}[plain]
    \titlepage
\end{frame}

% _____________________________________________________________________________

\begin{frame}{This course is open source}
    \begin{center}
        \githublink{\url{https://github.com/CExA-project/cexa-kokkos-tutorials}}
    \end{center}
\end{frame}

% _____________________________________________________________________________

\begin{frame}{Prerequisites}
    This course is intended for developers who have have followed the basic course of the tutorials

    \vspace{1em}

    \begin{block}{What you still need}
        \begin{itemize}
            \item Basic knowledge of C/C++
            \item Basic knowledge of parallel programming
            \item Basic knowledge of CMake
            \item Basic knowledge of a Linux environment
            \item Basic knowledge of Kokkos
        \end{itemize}
    \end{block}
\end{frame}

% _____________________________________________________________________________

\begin{frame}{Duration of the course}
    \begin{itemize}
        \item Course + practical work: full day
        \item Course + corrected exercise: half day
        \item Short version: 3 hours
    \end{itemize}
\end{frame}

\begin{frame}{Outline}
    \tableofcontents[hidesubsections]
\end{frame}

% _____________________________________________________________________________

\section{Profiling and debugging}

% _____________________________________________________________________________

\begin{frame}{Profiling and debugging tools at hand}
    \begin{columns}[T]
        \begin{column}{0.33\linewidth}
            Kokkos

            \begin{itemize}
                \item KokkosP interface
                \item Regions
            \end{itemize}

            \vspace{1em}

            \structure{Note:} Not actual tools, but used by tools of the two other categories
        \end{column}
        \begin{column}{0.33\linewidth}
            Kokkos tools

            \begin{itemize}
                \item Kernel timer
                \item Kernel logger
                \item Memory usage
                \item Memory events
                \item Space time stack
            \end{itemize}
        \end{column}
        \begin{column}{0.33\linewidth}
            Third-party tools

            \begin{itemize}
                \item VTune
                \item Nsight Systems
                \item Nsight Compute
                \item Tau
                \item Timemory
                \item Caliper
                \item HPCToolkit
            \end{itemize}
        \end{column}
    \end{columns}
\end{frame}

% _____________________________________________________________________________

\begin{frame}[fragile]{KokkosP interface}
    \begin{columns}
        \begin{column}{0.55\linewidth}
            \begin{minted}{C++}
                Kokkos::View<int *> view("view", N);

                Kokkos::parallel_for(
                    "compute",
                    N,
                    KOKKOS_LAMBDA(int i) {
                        view(i) = // ...
                    }
                );

                Kokkos::fence(
                    "wait for compute"
                );
            \end{minted}
        \end{column}
        \begin{column}{0.45\linewidth}
            \begin{itemize}
                \item Provided by Kokkos
                \item Hooks
                \begin{itemize}
                    \item Parallel constructs
                    \item Fences
                \end{itemize}
                \item Designed for other tools to use it
                \item Always available
                \item No overhead if no tools are used
            \end{itemize}
        \end{column}
    \end{columns}
\end{frame}

% _____________________________________________________________________________

\begin{frame}[fragile]{Regions}
    \begin{columns}
        \begin{column}{0.5\linewidth}
            \begin{minted}{C++}
                Kokkos::Profiling::pushRegion(
                    "computation"
                );

                // ...

                Kokkos::Profiling::popRegion();
            \end{minted}
        \end{column}
        \begin{column}{0.5\linewidth}
            \begin{itemize}
                \item Provided by Kokkos
                \item No specific header needed
                \item Namespace \texttt{Kokkos::Profiling}
                \item Set regions of interest in your code
                \item \texttt{pushRegion} to create a region
                \item \texttt{popRegion} to remove a region
                \item Regions can be nested
            \end{itemize}
        \end{column}
    \end{columns}
\end{frame}

% _____________________________________________________________________________

\begin{frame}{Kokkos tools}
    \begin{itemize}
        \item Library of the Kokkos ecosystem
        \item \githublink{\url{https://github.com/kokkos/kokkos-tools}}
        \item \doclink{\url{https://github.com/kokkos/kokkos-tools/wiki}}
        \item Has a different version number than Kokkos
        \item Should be built and installed somewhere (this requires to install Kokkos too)
        \item Use one tool at a time with the environment variable \texttt{KOKKOS\_TOOLS\_LIBS}

        Mind the \texttt{S}!
        \item Do not ship it within your program (this is a dev tool!)
    \end{itemize}
\end{frame}

% _____________________________________________________________________________

\begin{frame}[fragile]{Kernel timer for a basic profiling}
    \begin{columns}
        \begin{column}{0.6\linewidth}
            \begin{minted}[breakafter=/]{sh}
                export KOKKOS_TOOLS_LIBS=/absolute/path/to/libkp_kernel_timer.so

                ./my_program
            \end{minted}

            \usemintedstyle{vim}
            \begin{minted}{text}
                KokkosP: Simple Kernel Timer Library Initialized (sequence is 0, version: 20240906)
                KokkosP: Kernel timing written to /path/to/name_of_report.dat
            \end{minted}

            \begin{minted}{sh}
                kp_reader ./name_of_report.dat
            \end{minted}
        \end{column}
        \begin{column}{0.4\linewidth}
            \begin{itemize}
                \item Simple tool for a basic timing analysis
                \item Export environment variable to use the tool
                \item Run the program as usual
                \item Analyze the generated data with the provided \texttt{kp\_reader} program
            \end{itemize}
        \end{column}
    \end{columns}
\end{frame}

% _____________________________________________________________________________

\begin{frame}[fragile]{Kernel timer output}
    \begin{minted}{text}
         (Type)   Total Time, Call Count, Avg. Time per Call, %Total Time in Kernels, %Total Program Time
        -------------------------------------------------------------------------
        Regions:
        ...
        -------------------------------------------------------------------------
        Kernels:
        ...
        -------------------------------------------------------------------------
        Summary:

        Total Execution Time (incl. Kokkos + non-Kokkos):                   0.00038 seconds
        Total Time in Kokkos kernels:                                       0.00030 seconds
        -> Time outside Kokkos kernels:                                  0.00008 seconds
        -> Percentage in Kokkos kernels:                                   78.75 %
        Total Calls to Kokkos Kernels:                                            2

        -------------------------------------------------------------------------
    \end{minted}
\end{frame}

% _____________________________________________________________________________

\begin{frame}[fragile]{Kernel timer output (for regions)}
    \begin{minted}{text}
         (Type)   Total Time, Call Count, Avg. Time per Call, %Total Time in Kernels, %Total Program Time
        -------------------------------------------------------------------------

        Regions:

        - computation
        (REGION)   0.000349 1 0.000349 117.590361 92.599620
    \end{minted}

    \begin{itemize}
    	\item Regions are named after what you set in \texttt{pushRegion}
        \item Descending order
    	\item One set of two lines per region
    \end{itemize}
\end{frame}

% _____________________________________________________________________________

\begin{frame}[fragile]{Kernel timer output (for kernels)}
    \begin{minted}{text}
         (Type)   Total Time, Call Count, Avg. Time per Call, %Total Time in Kernels, %Total Program Time

        -------------------------------------------------------------------------
        Kernels:

        - compute
        (ParFor)   0.000282 1 0.000282 94.939759 74.762808
        - Kokkos::View::initialization [view] via memset
        (ParFor)   0.000015 1 0.000015 5.060241 3.984820
    \end{minted}

    \begin{itemize}
	    \item Kernels are named after what you set in parallel constructs (\texttt{parallel\_*})
        \item Some internal kernels are visible: Views initialization
        \item Descending order
	    \item One set of two lines per kernel
    \end{itemize}
\end{frame}

% _____________________________________________________________________________

\begin{frame}[fragile]{Kernel logger for a basic debugging}
    \begin{columns}
        \begin{column}{0.6\linewidth}
            \begin{minted}[breakafter=/]{sh}
                export KOKKOS_TOOLS_LIBS=/absolute/path/to/libkp_kernel_logger.so

                ./my_program
            \end{minted}

            \begin{minted}{text}
                KokkosP: Kernel Logger Library Initialized (sequence is 0, version: 20240906)
            \end{minted}
        \end{column}
        \begin{column}{0.4\linewidth}
            \begin{itemize}
                \item Simple tool for a basic timing analysis
                \item Export environment variable to use the tool
                \item Run the program as usual
            \end{itemize}
        \end{column}
    \end{columns}
\end{frame}

% _____________________________________________________________________________

\begin{frame}[fragile]{Kernel logger output}
    \begin{minted}[fontsize=\tiny]{text}
        KokkosP: Entering profiling region: computation
        KokkosP: Allocate<Cuda> name: view pointer: 0x7c67126aa000 size: 80000
        KokkosP: Executing fence on device 33554433 with unique execution identifier 0
        KokkosP: computation
        KokkosP:       SharedAllocationRecord<Kokkos::CudaSpace, void>::SharedAllocationRecord(): fence after copying header from HostSpace
        KokkosP: Execution of fence 0 is completed.
        KokkosP: Executing parallel-for kernel on device 33554433 with unique execution identifier 1
        KokkosP: computation
        KokkosP:       Kokkos::View::initialization [view] via memset
        KokkosP: Execution of kernel 1 is completed.
        KokkosP: Executing fence on device 33554433 with unique execution identifier 2
        KokkosP: computation

        ...

        KokkosP: Kokkos library finalization called.
    \end{minted}

    \begin{itemize}
        \item All activity listed (parallel constructs, fences, View creations...)
        \item Useful for tracking chaining of kernels
        \item Difficult to read when the kernels are asynchronous
    \end{itemize}
\end{frame}

% _____________________________________________________________________________

\begin{frame}[fragile]{NVTX connector for NVIDIA tools}
    \begin{columns}
        \begin{column}{0.6\linewidth}
            \begin{minted}[breakafter=/]{sh}
                export KOKKOS_TOOLS_LIBS=/absolute/path/to/libkp_nvtx_connector.so

                # nsight systems
                nsys profile -o report ./my_program
                nsys-ui report.nsys-rep

                # nsight compute
                ncu -o report ./my_program
                ncu-ui report.ncu-rep
            \end{minted}

            \begin{minted}{text}
                KokkosP: NVTX Analyzer Connector (sequence is 0, version: 20240906)
            \end{minted}
        \end{column}
        \begin{column}{0.4\linewidth}
            \begin{itemize}
                \item Simple tool to convert Kokkos named regions/kernels into NVTX regions for NVIDIA tools
                \begin{itemize}
                    \item Nsight Systems
                    \item Nsight Compute
                \end{itemize}
                \item Export environment variable to use the tool
                \item Run the program as usual
                \item Open the generated report in either tool
            \end{itemize}
        \end{column}
    \end{columns}
\end{frame}

\begin{frame}{Display NVTX name of kernels in Nsight Systems}
    \begin{columns}
        \begin{column}{0.5\linewidth}
            \begin{enumerate}
                \item Tools $\rightarrow$ Options
                \item Behavior $\rightarrow$ Report
                \item Set "Rename CUDA Kernels by NVTX" to \emph{Yes}
                \item Re-open the report
            \end{enumerate}
        \end{column}
        \begin{column}{0.5\linewidth}
            \includegraphics[width=\linewidth]{nsight_systems_options}
        \end{column}
    \end{columns}
\end{frame}

\begin{frame}{Nsight Systems overview}
    \begin{columns}
        \begin{column}{0.75\linewidth}
            \includegraphics[width=\linewidth]{nsight_systems_example}
        \end{column}
        \begin{column}{0.25\linewidth}
            \begin{itemize}
                \item Activity against execution time
                \item Cuda activity
                \begin{itemize}
                    \item Kernels
                    \item Memory
                \end{itemize}
                \item NVTX
                \begin{itemize}
                    \item Parallel constructs
                    \item Fences
                    \item Regions
                \end{itemize}
                \item Cuda API
            \end{itemize}
        \end{column}
    \end{columns}
\end{frame}

\begin{frame}{Nsight Compute overview}
    \begin{columns}
        \begin{column}{0.75\linewidth}
            \includegraphics[width=\linewidth]{nsight_compute_example}
        \end{column}
        \begin{column}{0.25\linewidth}
            \begin{itemize}
                \item Details of each kernel
                \item Many metrics
                \begin{itemize}
                    \item Speed of light
                    \item Occupancy
                    \item ...
                \end{itemize}
                \item Can give suggestions
                \item Corresponding PTX-source
            \end{itemize}
        \end{column}
    \end{columns}
\end{frame}

\begin{frame}{Apex autotuner}
    \begin{itemize}
        \item Apex is an autotuner which supports Kokkos natively
        \item Third-party that has to be installed
        \item \githublink{\url{https://github.com/UO-OACISS/apex}}
        \item \doclink{\url{https://uo-oaciss.github.io/apex/}}
        \item \doclink{\url{https://github.com/UO-OACISS/apex/wiki}}
    \end{itemize}
\end{frame}

\begin{frame}[fragile]{Use of Apex autotuner}
    \begin{columns}
        \begin{column}{0.5\linewidth}
            \begin{minted}[breakafter=/]{sh}
                /path/to/apex_exec --apex:kokkos --apex:cuda ./my_program
            \end{minted}

            \begin{minted}[fontsize=\tiny]{text}
                          (            )
                   (      )\ )      ( /(
                   )\    (()/( (    )\())
                ((((_)(   /(_)))\  ((_)\
                 )\ _ )\ (_)) ((_) __((_)
                 (_)_\(_)| _ \| __|\ \/ /
                  / _ \  |  _/| _|  >  <
                 /_/ \_\ |_|  |___|/_/\_\

                APEX Version: v2.7.1-593cd822-develop
                Built on: 14:22:04 Dec 10 2025 (Release)
                C++ Language Standard version : 201703
                GCC Compiler version : 13.3.0
                Configured features: Pthread, CUDA, PLUGINS

                Executing command line: ./my_program
            \end{minted}
        \end{column}
        \begin{column}{0.5\linewidth}
            \begin{itemize}
                \item Analysis of Kokkos parallel constructs
                \item Analysis of data movements
                \item Analysis of task dependencies
            \end{itemize}
        \end{column}
    \end{columns}
\end{frame}

% _____________________________________________________________________________

\section{Subviews}

% _____________________________________________________________________________

\section{Atomics}

% _____________________________________________________________________________

\section{Layouts}

% _____________________________________________________________________________

\section{Subviews}

% _____________________________________________________________________________

\section{Scatter Views}

\end{document}
